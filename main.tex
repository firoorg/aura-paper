\documentclass{llncs}

\pagestyle{plain}

\usepackage{amsmath,amssymb,amsfonts}
\usepackage{bookmark}

\newcommand{\G}{\mathbb{G}}
\newcommand{\F}{\mathbb{F}}
\newcommand{\func}[1]{\mathsf{#1}}

\begin{document}

\title{Aura: private voting with reduced trust on tallying authorities}
\author{Aram Jivanyan\inst{1,2}\thanks{Corresponding author: \email{aram@firo.org}} \and Aaron Feickert\inst{3}}
\institute{Firo \and Yerevan State University \and Cypher Stack}
\maketitle

\begin{abstract}
	Electronic voting has long been an area of active and challenging research.
	Security properties relevant to physical voting in elections with a variety of threat models and priorities are often difficult to reproduce in cryptographic systems and protocols.
    Even in voting systems where ballot contents are private and results are verifiable, ballot anonymity is often offloaded to requirements of trusted parties.
	Here we introduce Aura, an election protocol that reduces trust on tallying authorities and organizers while ensuring voter privacy.
	Ballots in Aura are dissociated from voter identity cryptographically and use verifiable encryption and threshold decryption to mitigate trust in tallying authorities
    Aura requires no trusted setups for cryptographic primitives and uses efficient proving systems to reduce computation and communication complexity.
	These properties make Aura a competitive candidate for use in a variety of applications where verifiable trust minimization is desirable or necessary.
\end{abstract}


\section{Introduction}

Electronic voting poses unique and systemic challenges in research, development, implementation, and deployment.
Verifiable electronic voting, where a public election record is employed for transparency and auditability, has inherently different trust and security requirements than other legacy techniques; in this case, ballot and election properties and tabulation methods must be secured cryptographically in order to achieve the required goals of a particular application.

Requirements, risks, and threat models in elections are complex and varied.
Ballot anonymity is often required and reasonably guaranteed in physical elections, where ballots contain no identifying information about the voter at the time of tallying.
Avoidance of voter coercion and bribery may also be important in major elections; a voter entering a voting booth alone where photography is prohibited can prevent this in practice, but this may not be case if the election is conducted online.


\subsection{Requirements}

Properties and requirements on voting protocols have long been the subject of interesting and evolving research, but as yet there does not appear to be a universal set of guidelines by which to analyze such constructions.

Informally, we require the following properties:
\begin{itemize}
    \item \textbf{Public parameters}: Aside from election-specific trust requirements, all cryptographic constructions must use publicly-verifiable parameters.
    \item \textbf{Correctness}: A voter authorized for an election can cast a ballot that is included in the election result.
    \item \textbf{Universal verifiability}: Any observer can verify that all valid ballots are included in the final result, and that the result correctly represents only those ballots.
    \item \textbf{Vote privacy}: It is not possible for an observer to determine the vote associated with a valid ballot.
    \item \textbf{Voter anonymity}: It is not possible for an observer to determine the voter associated with a valid ballot, or if a particular voter voted at all.
    \item \textbf{Coercion resistance}: It is possible for a voter to privately cast multiple ballots that each invalidate any previous ballots.
\end{itemize}

Coercion resistance assumes the possibility that a voter may be bribed or coerced into voting a particular choice, but is outside adversarial influence at a later time prior to the election ending.
It is often related to the idea of a receipt-free election, where a voter is not able to provide evidence of its vote to a third party at any time; while Aura does not have this property, we consider the listed form of coercion resistance to be useful nonetheless.


\subsection{Prior work}

There is a large and growing body of research over several decades relating to security models and instantiations of electronic voting protocols using a variety of cryptographic techniques, but we do not attempt to provide a comprehensive review here.

Arguably one of the most relevant comparisons to our current work is Helios, a popular deployed protocol for so-called ``boardroom" elections where many risks relevant to large-scale public elections are not present.
The original Helios protocol \cite{helios} relies heavily on talliers, election organizers, and a central server; ballots are publicly linked to voter identity, and talliers act as a mixnet to shuffle ballots prior to decryption.
Later work proposed an informally-described protocol update to Helios \cite{helios2} that replaces expensive verifiable shuffling with homomorphic ballot decryption and a set of proofs of ballot validity; however, individual ballots are still linked to voter identity.
The research of \cite{cortier} introduces a straightforward verifiable ElGamal threshold cryptosystem for talliers that does not require a trusted dealer, and augments Helios to include this; however, the method provided is vulnerable to key cancellation and provides no particular guarantees on key validity.
Another relevant comparison is ElectionGuard, a well-specified protocol for verifiable elections \cite{electionguard} that includes more robust verifiable key generation and decryption.
However, it does not provide any particular verifiable guarantees on ballot anonymity, relying on election administrators to assert voter eligibility and decouple voter identity from ballot data.
Additionally, although it provides an option for ballot spoiling, this requires individual decryption of such ballots for verification.

More recent investigations introduce complete voter privacy with different trust requirements, primarily using combinations of encrypted ballots and general zero-knowledge proving systems, to dissociate ballots from voter identity.
For example, \cite{dimitriou} uses a zk-SNARK construction to anonymize ballots, and relies on organizer-supplied token randomizers as a form of coercion resistance; however, soundness and voter anonymity are compromised in the case of a malicious organizer producing the proving system common reference string.
In Vote-SAVER \cite{saver}, voter anonymity is similarly provided by a zk-SNARK construction, and coercion resistance is achieved by having untrusted third parties conduct provable re-randomization; however, this crucially relies on proving system malleability, and therefore is currently limited (to our knowledge) to proving systems where soundness depends on a trusted organizer to produce a non-malicious common reference string.
More recent work like Kryvos \cite{kryvos} examines more complex voting methods and adds partial or full hiding of tally details, but soundness depends on trusted organizers and proofs are large.


\subsection{Contribution}

Aura presents a protocol combining several useful properties that improve on earlier work.

First, we minimize the trust on election participants, including tally authorities with the joint capability to decrypt ballot results.
In Aura, all cryptographic components may be instantiated with public verifiable parameters.
Keys used by voters can be generated by voters themselves, and the key used for decrypting election results is constructed by tally authorities in a distributed and verifiable manner that does not require a trusted dealer.
Ballots are dissociated from voter identity using voter-produced provable re-randomization and a set membership proof, and ballot validity is asserted by a combination of verifiable ElGamal encryption and a bit vector proving system.
Even in the case of collusion between talliers (and even organizers) to decrypt individual ballots, voter anonymity is perfectly retained; and while multiple vote attempts by a voter can be reliably detected, this process occurs after the close of the election, and allows for safer mitigation of voter coercion by permitting such a voter to invalidate a coerced ballot anonymously and without revealing its contents.
The careful use of proving systems for ballot anonymity and coercion resistance additionally prevents votes cast in replay attacks from succeeding with either identical or malleated ballot data \cite{replay}.

Aura uses constructions supporting efficient operations.
The one-of-many proving system used to assert voter anonymity supports batch verification that greatly reduces the marginal complexity of verification, and scales extremely well in proof size even with a large number of voters.
Further, a single bit vector commitment proving system is used to assert that a set of vote ciphertexts are valid, both with valid ElGamal vote messages and the overall number of choices selected by a voter; this proving system also supports batch verification and scales more efficiently than previous work, while remaining flexible for single- and multi-choice election rules.

While we use well-studied techniques and cryptographic components to build Aura, we stress that the overall protocol analysis is informal, and we defer a formal security model and proofs to future work.


\section{Cryptographic primitives}

In this section, we describe the cryptographic constructions required for the Aura election protocol.
Throughout these descriptions, let $\G$ be a prime-order group where the discrete logarithm and decisional Diffie-Hellman problems are hard, and let $\F$ be its scalar field.


\subsection{Distributed verifiable threshold ElGamal encryption}

Aura requires a distributed verifiable threshold ElGamal cryptosystem, where a cohort of designated parties is required to decrypt messages.
In such a construction, we require that key generation be fully distributed with no trusted parties.
Further, the validity of key generation, encryption, and decryption must be verifiable.

The construction we describe here is based on that of \cite{cortier}, which describes a distributed threshold design intended for use in Helios.
However, that construction is vulnerable to key cancellation attacks, does not assert proper joint key representation, and uses verification keys that (if maliciously crafted) do not allow for publicly-verifiable decryption.
Further, the design is generic to support arbitrary group elements as messages, which is not secure in general \cite{boneh}; while its overlying protocol does not fall victim to this problem by the nature of its construction, the general design is vulnerable.
We modify the design to address these shortcomings, specify abort points in the protocol, and indicate simplifications where possible.

Let $pp_{\text{enc}} = (\G, \F, G, \{H_i\}_{i=0}^{k-1}, k, t, \nu)$ be the public parameters for the construction, where $G, \{H_i\}_{i=0}^{k-1} \in \G$ are independent generators, $k > 0$ is the number of valid message generators, $t$ is the threshold of keyholders required for decryption, and $\nu$ is the total number of keyholders (so $1 \leq t \leq \nu)$.
We assume that $pp_{\text{enc}}$ is available to all algorithms, which we describe now:
\begin{itemize}
    \item $\func{KeyGen}(\alpha) \mapsto (Y_\alpha, \Pi_\alpha^{\text{key}})$: The function takes as input a player index $1 \leq \alpha \leq \nu$.
    It does the following:
    \begin{enumerate}
        \item Chooses a set $\{a_{\alpha,j}\}_{j=0}^{t-1} \subset \F$ of scalars uniformly at random, and defines the polynomial \[ f_\alpha(x) = \sum_{j=0}^{t-1} a_{\alpha,j}x^j\] and vector $C_\alpha = \{C_{\alpha,j}\}_{j=0}^{t-1} = \{a_{\alpha,j}G\}_{j=0}^{t-1}$ using these values.
        \item Produces a proof of representation $\Pi_\alpha^{\text{rep}} = \func{RepProve}(G, C_{\alpha,0} ; a_{\alpha,0})$, and sends the tuple $(C_\alpha, \Pi_\alpha^{\text{rep}})$ to all other players.
        \item On receipt of such a tuple $(C_\beta, \Pi_\beta^{\text{rep}})$ from another player $\beta$, verifies that $\func{RepVerify}(\Pi_\beta^{\text{rep}}, G, C_\beta) = 1$, and aborts otherwise.
        \item For each $1 \leq \beta \leq \nu$, computes a value $y_{\alpha, \beta} = f_\alpha(\beta)$ and sends it to player $\beta$.
        \item On receipt of such a value $y_{\beta,\alpha}$ from another player $\beta$, checks that \[ \sum_{j=0}^{t-1} C_{\beta,j} = y_{\beta,\alpha}G \] and aborts otherwise.
        \item Computes its private key share \[ y_\alpha = \sum_{\beta=1}^{\nu} y_{\beta,\alpha} \] and public key share $Y_\alpha = y_\alpha G$ and public group key \[ Y = \sum_{\beta=1}^{\nu} C_{\beta,0}. \]
        \item Produces a proof of representation $\Pi_\alpha^{\text{key}} = \func{RepProve}(G, Y_\alpha ; y_\alpha)$.
    \end{enumerate}
    The function outputs $(Y_\alpha, \Pi_\alpha^{\text{key}})$.
    
    \item $\func{VerifyKeyGen}(\{Y_\alpha, \Pi_\alpha^{\text{key}}\}_{\alpha=1}^{\nu}) \mapsto Y$: The function takes as input a set of key shares and proofs from a set of $\nu$ players.
    It does the following:
    \begin{enumerate}
        \item For each $1 \leq \alpha \leq \nu$, checks that $\func{RepVerify}(\Pi_\alpha^{\text{key}}, G, Y_\alpha) = 1$, and aborts otherwise.
        \item Sets $Y = \sum_{\alpha=1}^{\nu} Y_\alpha$.
    \end{enumerate}
    The function outputs $Y$.
    
    \item $\func{Encrypt}(m, i, Y) \mapsto (D, E, \Pi_{\text{enc}})$: The function takes as input a message $m \in \F$, a message generator index $0 \leq i < k$, and a public key $Y$.
    It does the following:
    \begin{enumerate}
        \item Chooses a nonce $r \in \F$ uniformly at random.
        \item Sets $D = rG$ and $E = rY + mH_i$.
        \item Produces a proof of encryption: \[ \Pi_{\text{enc}} = \func{EncValProve}(G, Y, H_i, D, E ; (r, m)) \]
    \end{enumerate}
    The function outputs $(D, E, \Pi_{\text{enc}})$.
    
    \item $\func{VerifyEncrypt}(Y, i, D, E, \Pi_{\text{enc}}) \mapsto \{0, 1\}$: The function takes as input an ElGamal public key $Y$, message generator index $0 \leq i < k$, ElGamal ciphertext $(D, E)$, and a proof of encryption.
    If \[ \func{EncValVerify}(\Pi_{\text{enc}}, G, Y, H_i, D, E) = 1 \] it outputs $1$; otherwise, it outputs $0$.
    
    \item $\func{PartialDecrypt}(y_\alpha, D, E) \mapsto (R_\alpha, \Pi_\alpha^{\text{dec}})$: The function takes as input a private key share $y_\alpha$ and ElGamal ciphertext $(D, E)$.
    It does the following:
    \begin{enumerate}
        \item Computes $R_\alpha = y_\alpha D$.
        \item Produces a proof of discrete logarithm equality: \[ \Pi_\alpha^{\text{dec}} = \func{EqProve}(D, G, R_\alpha, y_\alpha G ; y_\alpha) \]
    \end{enumerate}
    The function outputs $(R_\alpha, \Pi_\alpha^{\text{dec}})$.
    
    \item $\func{VerifyDecrypt}(D, E, \{j, Y_j, R_j, \Pi_j^{\text{dec}}\}_{j=1}^t) \mapsto m$: The function takes as input ElGamal ciphertext $(D, E)$, a threshold set of $t$ player indices, corresponding public key shares, and associated partial decryption data.
    We note that for the sake of notation convenience, the set of players is reindexed here; in practice, any threshold of players may be used with their corresponding indices.
    It does the following:
    \begin{enumerate}
        \item For each $1 \leq j \leq t$, checks that $\func{EqVerify}(\Pi_j^{\text{dec}}, D, G, R_j, Y_j) = 1$, and aborts otherwise.
        \item For each $1 \leq j \leq t$, computes the corresponding Lagrange coefficient: \[ \lambda_j = \prod_{i=1, i \neq j}^t \frac{i}{i - j} \]
        \item Computes the following: \[ M = E - \sum_{j=1}^t \lambda_j R_j \]
        \item Uses brute force (or another appropriate computational method) to find $m \in \F$ such that $mH = M$.
    \end{enumerate}
    The function outputs $m$.
\end{itemize}


\subsection{Proving systems}

We require several proving systems for use in Aura.
Each can be instantiated non-interactively, either by instantiations cited, or using standard Schnorr-type representation proof techniques.
Each is provable to be complete, special sound, and special honest-verifier zero knowledge.
For each proving system, we list the public parameters, relevant relation, and prover and verifier functions; we omit the specific instantiations.


\subsubsection{Bit vector commitment proving system}

This proving system asserts that a given group element is a Pedersen vector commitment to elements in the set $\{0,1\}$ whose sum is a specified value.
The public parameters are $pp_{\text{bit}} = \left( \G, \F, w, k, \{G_i\}_{i=0}^{k-1}, H \right)$, where $w, k > 0$ and $\{G_i\}_{i=0}^{k-1}, H \in \G$ are independent generators.
The relation is the following:
\begin{multline*}
\mathcal{R}_{\text{bit}} = \left\{ pp_{\text{bit}}, B \in \G ; \{b_i\}_{i=0}^{k-1}, r \in \F : B = rH + \sum_{i=0}^{k-1} b_i G_i, \right. \\
\left. b_i \in \{0,1\} \forall i \in [0,k), \sum_{i=0}^{k-1} b_i = w \right\}
\end{multline*}
The relevant algorithms are the following:
\begin{itemize}
	\item $\func{BitProve}\left( B ; \{b_i\}_{i=0}^{k-1}, r \right) \mapsto \Pi_{\text{bit}}$
	\item $\func{BitVerify}\left( \Pi_{\text{bit}}, B \right) \mapsto \{0, 1\}$
\end{itemize}
A simple generalization of an existing proving system by Bootle \textit{et al.} may be used as an instantiation of the required proving system \cite{bootle}.
For completeness, we include a full description of the generalization in Appendix \ref{app:bit}.


\subsubsection{Commitment set proving system}

This proving system asserts that some group element in a given set is, when offset by another group element, a Pedersen commitment to zero.
The public parameteres are $pp_{\text{set}} = (\G, \F, G, H, n, m)$, where $n, m > 1$ and $G, H$ are independent generators.
Let $N = n^m$.
The relation is the following:
\[ \mathcal{R}_{\text{set}} = \left\{ pp_{\text{set}}, \{C_i\}_{i=0}^{N-1}, C' \in \G ; l \in [0,N), r \in \F : C_l - C' = rH \right\} \]
The relevant algorithms are the following:
\begin{itemize}
	\item $\func{SetProve}\left( \{C_i\}_{i=0}^{N-1}, C' ; l, r \right) \mapsto \Pi_{\text{set}}$
	\item $\func{SetVerify}\left( \Pi_{\text{set}}, \{C_i\}_{i=0}^{N-1}, C' \right) \mapsto \{0, 1\}$
\end{itemize}
The proving system in \cite{bootle}, with a simple modification as done in \cite{spark}, may be used for this purpose.



\subsubsection{Representation proving system}

This proving system asserts knowledge of a group element representation.
The public parameters are $pp_{\text{rep}} = (\G, \F)$.
The relation is the following:
\[ \mathcal{R}_{\text{rep}} = \{ pp_{\text{rep}}, \{G_i\}_{i=0}^{n-1}, Y ; \{y_i\}_{i=0}^{n-1} : Y = \sum_{i=0}^{n-1} y_i G_i \} \]
The relevant algorithms are the following:
\begin{itemize}
    \item $\func{RepProve}(\{G_i\}_{i=0}^{n-1}, Y ; \{y_i\}_{i=0}^{n-1}) \mapsto \Pi_{\text{rep}}$
    \item $\func{RepVerify}(\Pi_{\text{rep}}, \{G_i\}_{i=0}^{n-1}, Y) \mapsto \{0, 1\}$
\end{itemize}


\subsubsection{Encryption validity proving system}

This proving system asserts a valid ElGamal encryption using a specific representation assertion.
The public parameters are $pp_{\text{val}} = (\G, \F)$.
The relation is the following:
\[ \mathcal{R}_{\text{val}} = \{ pp_{\text{enc}}, G, Y, H, D, E ; (r, m) : D = rG, E = mY + rH \} \]
The relevant algorithms are the following:
\begin{itemize}
    \item $\func{EncValProve}(G, Y, H, D, E ; r, m) \mapsto \Pi_{\text{enc}}$
    \item $\func{EncValVerify}(\Pi_{\text{enc}}, G, Y, H, D, E) \mapsto \{0, 1\}$
\end{itemize}


\subsubsection{Serial validity proving system}

This proving system asserts a valid ElGamal encryption using a specific representation assertion matches a particular partial commitment opening.
The public parameters are $pp_{\text{ser}} = (\G, \F)$.
The relation is the following:
\begin{multline*}
\mathcal{R}_{\text{ser}} = \left\{ pp_{\text{ser}}, F, G, H, Y, C, D, E ; (s, r, r') : \right. \\
\left. C = sG + rH, D = r'G, E = sF + r'Y \right\}
\end{multline*}
The relevant algorithms are the following:
\begin{itemize}
    \item $\func{SerValProve}(F, G, H, Y, C, D, E ; s, r, r') \mapsto \Pi_{\text{ser}}$
    \item $\func{SerValVerify}(\Pi_{\text{ser}}, F, G, H, Y, C, D, E) \mapsto \{0, 1\}$
\end{itemize}


\subsubsection{Discrete logarithm equality proving system}

This proving system asserts two group elements share the same discrete logarithm with respect to specified generators.
The public parameters are $pp_{\text{eq}} = (\G, \F)$.
The relation is the following:
\[ \mathcal{R}_{\text{eq}} = \{ pp_{\text{eq}}, G, H, Y, Y' ; y : Y = yG, Y' = yH \} \]
The relevant algorithms are the following:
\begin{itemize}
    \item $\func{EqProve}(G, H, Y, Y' ; y) \mapsto \Pi_{\text{eq}}$
    \item $\func{EqVerify}(\Pi_{\text{eq}}, G, H, Y, Y') \mapsto \{0, 1\}$
\end{itemize}


\subsection{Unforgeable signature scheme}

In Aura, different types of participants submit messages to a public bulletin board.
For some of the messages, observers must verify their authenticity in order to assert they are created by the claimed entity.
For other messages, this property is not required (or even desired).
We therefore assume the existence of an unforgeable signature scheme on arbitrary messages that can be bound to contexts to mitigate replay attacks.
Constructions like context-prefixed Schnorr digital signatures may be used for this purpose.
In the protocol, we describe which entities are assumed to possess signing and verification keys for this signature scheme.


\section{Protocol}


\subsection{Overview}

There are several types of entities in Aura that interact during the election process.
Organizers set protocol parameters for elections, voters, and talliers.
Voters cast ballots in elections.
Talliers collaboratively compute and publish results at the end of elections.
Verifiers assert that elections and results are valid.

We assume a public bulletin board $\mathcal{B}$ is used to store election data.
The instantiation of $\mathcal{B}$ is especially suited for a blockchain-type construction for which modification or erasure of posted data is computationally infeasible.


\subsection{Algorithms}

An election consists of several steps, represented by algorithms that we describe in detail here.
We assume that the organizer, the talliers, and all voters possess signing keys (with corresponding verification keys) for the unforgeable signature scheme, which can be used to sign and verify arbitrary messages to authenticate them.
The distribution of such keys is outside the scope of this protocol.


\subsubsection{\texorpdfstring{$\func{SetupElection}$}{SetupElection}}

The organizer does the following:
\begin{enumerate}
\item Chooses a unique election identifier $\mathbb{I} \in \{0,1\}^*$, and prepares parameter $m_{\text{elec}}$ as a human-readable description of the election, which may include auxiliary information for voters as necessary by election rules.
\item Selects parameter $k > 0$ as the number of candidates or choices in the election and, for each $i \in [0,k)$, produces a pair $(i,m_i)$, where $m_i$ is a human-readable description of choice $i$.
\item Selects parameters $k_{\text{min}}$ and $k_{\text{max}}$ corresponding (respectively) to the minimum and maximum number of choices a voter may make; we require that $1 \leq k_{\text{min}} \leq k_{\text{max}} \leq k$.
For convenience, let $k' = k + k_{\text{max}} - k_{\text{min}}$.
\item For each $i \in [0,k')$, samples a generator $H_i \in \G$ uniformly at random in a publicly-verifiable way.
\item Samples group generators $F, G, H \in \G$ uniformly at random in a publicly-verifiable way.
\item Prepares a list $L_{\text{voters}}$ of the $N_{\text{voters}}$ voter verification keys corresponding to authorized voters in the election, and lets $n,m > 1$ such that $N_{\text{voters}} = n^m$.
\item Prepares a list $L_{\text{tally}}$ of the $N_{\text{tally}} > 0$ tallier verification keys corresponding to the authorized talliers in the election, and a threshold $1 \leq t \leq N_{\text{tally}}$ of talliers required for result decryption.
\item Prepares the public parameters for required underlying cryptographic constructions:
\begin{itemize}
\item Samples a prime-order group $\G$ with a scalar field $\F$.
\item Sets $pp_{\text{enc}} = (\G, \F, \{H_i\}_{i=0}^{k'-1}, k', t, N_{\text{tally}})$ as the parameters for a distributed verifiable ElGamal encryption system.
\item Sets $pp_{\text{bit}} = (\G, \F, k_{\text{max}}, k', \{H_i\}_{i=0}^{k'-1}, -)$ as the parameters for a bit vector commitment proving system, where we leave the final parameter undefined (to be set at a later step).
\item Sets $pp_{\text{set}} = (\G, \F, G, H, n, m)$ as the parameters for a commitment set proving system.
\item Sets $pp_{\text{rep}} = (\G, \F)$ as the parameters for a representation proving system.
\item Sets $pp_{\text{val}} = (\G, \F)$ as the parameters for an encryption validity proving system.
\item Sets $pp_{\text{ser}} = (\G, \F)$ as the parameters for a serial validity proving system.
\item Sets $pp_{\text{eq}} = (\G, \F)$ as the parameters for a discrete logarithm equality proving system.
\end{itemize}
\item Assembles the protocol public parameters
\begin{multline*}
    pp = (\mathbb{I}, m_{\text{elec}}, k, \{i,m_i\}_{i=0}^{k-1}, k_{\text{min}}, k_{\text{max}}, \G, \F, \{H_i\}_{i=0}^{k'-1}, F, G, H, \\
    L_{\text{voters}}, N_{\text{voters}}, n, m, L_{\text{tally}}, N_{\text{tally}}, t)
\end{multline*}
and posts them to $\mathcal{B}$ as an authenticated message signed with the organizer signing key.
\end{enumerate}

The public parameters $pp$ are assumed to be available to all participants and algorithms; further, all other subprotocol public parameters can be deterministically produced from $pp$.


\subsubsection{\texorpdfstring{$\func{SetupTally}$}{SetupTally}}

Each tallier with index $1 \leq \alpha \leq N_{\text{tally}}$ does the following:
\begin{enumerate}
\item Verifies the authenticated organizer message on $\mathcal{B}$ containing $pp$, and checks the validity of the parameters.
\item Runs $\func{KeyGen}(\alpha) \mapsto (Y_\alpha, \Pi_\alpha^{\text{key}})$ interactively with the other talliers.
\item Posts the values $(\alpha, Y_\alpha, \Pi_\alpha^{\text{key}})$ to $\mathcal{B}$ as an authenticated message signed with its tallier signing key from $L_{\text{tally}}$.
\end{enumerate}


\subsubsection{\texorpdfstring{$\func{SetupVoter}$}{SetupVoter}}

Each voter with index $0 \leq i < N_{\text{voters}}$ does the following:
\begin{enumerate}
    \item Verifies the authenticated organizer message on $\mathcal{B}$ containing $pp$, and checks the validity of the parameters.
    \item Selects $s_i, r_i \in \F$ uniformly at random, and privately stores these values.
    \item Computes a ballot key $C_i = s_i G + r_i H$.
    \item Generates a proof of representation $\func{RepProve}(\{G, H\}, C_i ; \{s, r\}) \mapsto \Pi_{\text{rep},i}$.
    \item Posts $(i, C_i, \Pi_{\text{rep},i})$ to $\mathcal{B}$ as an authenticated message signed with its voter signing key from $L_{\text{voters}}$.
\end{enumerate}
We note that it is safe for a voter to reuse their ballot key across multiple elections.


\subsubsection{\texorpdfstring{$\func{VerifySetup}$}{VerifySetup}}

The verifier does the following:
\begin{enumerate}
    \item Verifies the unique authenticated organizer message on $\mathcal{B}$ containing $pp$, and checks the validity of the parameters.
    \item For each $1 \leq \alpha \leq N_{\text{tally}}$, verifies the unique authenticated tallier message on $\mathcal{B}$ containing $(\alpha, Y_\alpha, \Pi_\alpha^{\text{key}})$ using the corresponding verification key from $L_{\text{tally}}$.
    \item Verifies the tally keys by running $\func{VerifyKeyGen}(\{Y_\alpha, \Pi_\alpha^{\text{key}}\}_{\alpha=1}^{\nu}) \mapsto Y$.
    \item For each $0 \leq i < N_{\text{voters}}$, verifies the unique authenticated voter message on $\mathcal{B}$ containing $(i, C_i, \Pi_{\text{rep},i})$, and verifies the ballot key by checking that $\func{RepVerify}(\Pi_{\text{rep},i}, \{G, H\}, C_i\} \mapsto 1$.
\end{enumerate}
At this point, all participants use $Y$ as the undetermined parameter in $pp_{\text{bit}}$.


\subsubsection{\texorpdfstring{$\func{Vote}$}{Vote}}

Each voter with index $0 \leq i < N_{\text{voters}}$ does the following:
\begin{enumerate}
    \item Constructs a vector $c_i = (c_{i,j})_{j=0}^{k-1}$ representing its choices among the $k$ options, where
    \begin{displaymath}
        c_{i,j} = \left\{
        \begin{array}{ll}
            1 & \text{if the voter chooses option } j \\
            0 & \text{otherwise}
        \end{array}
        \right.
    \end{displaymath}
    and $k_{\text{min}} \leq \sum_{j=0}^{k-1} c_{i,j} \leq k_{\text{max}}$.
    \item For $j \in [0,k)$, encrypts each choice by setting \[ (D_{i,j}, E_{i,j}, \Pi_{\text{enc},i,j}) = \func{Encrypt}(c_{i,j}, j, Y). \]
    \item For $j \in [k,k')$, extends the vector $c_i$ by setting
    \begin{displaymath}
        c_{i,j} = \left\{
        \begin{array}{ll}
            1 & \text{if } j < k + k_{\text{max}} - \sum_{j=0}^{k-1} c_{i,j} \\
            0 & \text{otherwise}
        \end{array}
        \right.
    \end{displaymath}
    for padding purposes, and computes encryptions \[ (D_{i,j}, E_{i,j}, \Pi_{\text{enc},i,j}) = \func{Encrypt}(c_{i,j}, j, Y). \]
    \item Computes a bit vector commitment proof \[ \Pi_{\text{bit},i} = \func{BitProve}\left( \sum_{j=0}^{k'-1} E_{i,j} ; \{c_{i,j}\}_{j=0}^{k'-1}, r \right), \] where $r$ is the sum of all nonces used in encryption proofs for $j \in [0,k'-1)$.
    \item Chooses a nonce $r_i' \in \F$ uniformly at random, and computes the serial offset $C_i' = s_i G + r_i' H$.
    \item Encrypts the ballot serial number by choosing a nonce $r_i'' \in \F$ uniformly at random and computing $D_i' = r_i'' G$ and $E_i' = s_i F + r_i'' Y$.
    \item Assembles $\overline{C}$ to be the set of all voter commitments $\{C_i\}$ corresponding to voter verification keys in $L_{\text{voters}}$, and generates a commitment set proof
    \[ \Pi_{\text{set},i} = \func{SetProve}\left(\overline{C}, C_i' ; l_i, r_i - r_i' \right) \]
    where $\overline{C}_{l_i} = C_i$.
    \item Assembles a ballot tuple:
    \[ B_i = \left( pp, (D_{i,j}, E_{i,j}, \Pi_{\text{enc},i,j})_{j=0}^{k'-1}, \Pi_{\text{bit},i}, C_i', D_i', E_i', \Pi_{\text{set},i} \right) \]
    \item Generates a proof of serial number validity
    \[ \Pi_{\text{ser},i} = \func{SerValProve}(F, G, H, Y, C_i', D_i', E_i' ; s_i, r_i', r_i'') \]
    that binds $B_i$ to its initial transcript.
    \item Posts the ballot tuple $B_i$ and binding proof $\Pi_{\text{ser},i}$ to $\mathcal{B}$.
\end{enumerate}

If the voter is coerced or bribed to submit a ballot of an adversary's choice, the voter may cast another ballot once outside of the adversary's influence by repeating these steps.
As shown below, such duplicate ballots will be accepted to $\mathcal{B}$, but will be excluded from the final tally except for the last such ballot cast by the voter.
This is intended to provide a weak form of coercion resistance.


\subsubsection{\texorpdfstring{$\func{VerifyBallot}$}{VerifyBallot}}

Given a semantically-correct ballot (without explicit voter index $i$) of the form
\[ B = \left( (D_j, E_j, \Pi_{\text{enc},j})_{j=0}^{k'-1}, \Pi_{\text{bit}}, C', D', E', \Pi_{\text{set}} \right), \]
any verifier does the following:
\begin{enumerate}
    \item Checks that $\func{SerValVerify}(\Pi_{\text{ser}}, F, G, H, Y, C', D', E') \mapsto 1$ using $B$ as a transcript binding, and aborts otherwise.
    \item For each $j \in [0,k')$, checks that $\func{VerifyEncrypt}(Y, j, D_j, E_j, \Pi_{\text{enc},j}) \mapsto 1$, and aborts otherwise.
    \item Checks that \[ \func{BitVerify}\left( \Pi_{\text{bit}}, \sum_{j=0}^{k'-1} E_j \right) \mapsto 1, \] and aborts otherwise.
    \item Assembles the set $\overline{C}$ as in $\func{Vote}$, checks that $\func{SetVerify}(\Pi_{\text{set}}, \overline{C}, C') \mapsto 1$, and aborts otherwise.
\end{enumerate}


\subsubsection{\texorpdfstring{$\func{Tally}$}{Tally}}

The talliers first verifiably decrypt all ballot serial numbers in order to complete the assertion of their validity and discard (for coercion-resistance purposes) recast ballots by common anonymized voters.
Assume a set of $t$ talliers indexed $1 \leq j \leq t$.
Each such tallier does the following for each valid ballot $i$ appearing on $\mathcal{B}$:
\begin{enumerate}
    \item Runs $\func{PartialDecrypt}(y_j, D_i', E_i') \mapsto (R_{\text{ser},i,j}, \Pi_{\text{ser},i,j})$, and posts this tuple to $\mathcal{B}$ as an authenticated message signed with its tallier signing key from $L_{\text{tally}}$.
    \item After receiving all such partial decryptions from the threshold cohort and verifying the authenticated messages, partially (without attempting to brute-force the final decryption) runs \[ \func{VerifyDecrypt}(D_i', E_i', \{j, Y_j, R_{\text{ser},i,j}^{\text{dec}}, \Pi_{\text{ser},i,j}\}_{j=1}^t) \] to obtain a serial number public key $S_i \in \G$.
    \item Verifies the signature on the ballot $i$ using $S_i$ as the verification public key (against generator $F$).
    \item If $S_i$ appears with any other valid ballot, discard all but the most recent such ballot, according to bulletin board ordering.
\end{enumerate}

At this point, let there be $N_{\text{valid}}$ remaining valid ballots, indexed by $i$.
The talliers now verifiably produce the tally of all $N_{\text{valid}}$ such ballots.
Each tallier with index $1 \leq j \leq t$ does the following:
\begin{enumerate}
    \item For each $l \in [0,k)$, computes the ballot sums for choice $l$ by setting \[ \overline{D}_l = \sum_{i=0}^{N_{\text{valid}}-1} D_{i,l} \] and \[ \overline{E}_l = \sum_{i=0}^{N_{\text{valid}}-1} E_{i,l}, \] and partially decrypting the sums:
    \[ \func{PartialDecrypt}(y_j, \overline{D}_l, \overline{E}_l) \mapsto (R_{l,j},\Pi_{l,j}^{\text{dec}}) \]
    \item Posts the set of tuples $\{(R_{l,j},\Pi_{l,j}^{\text{dec}})\}_{l=0}^{k-1}$ to $\mathcal{B}$ as an authenticated message signed with its tallier signing key from $L_{\text{tally}}$.
\end{enumerate}


\subsubsection{\texorpdfstring{$\func{VerifyTally}$}{VerifyTally}}

Any verifier checks the authenticity of all tallier messages posted from $\func{Tally}$ and does the following:
\begin{enumerate}
    \item For each valid ballot $i$ appearing on $\mathcal{B}$:
    \begin{enumerate}
        \item Partially (without attempting to brute-force the final decryption) runs \[ \func{VerifyDecrypt}(D_i', E_i', \{j, Y_j, R_{\text{ser},i,j}^{\text{dec}}, \Pi_{\text{ser},i,j}\}_{j=1}^t) \] to obtain a serial number public key $S_i \in \G$, and aborts if this fails.
        \item Verifies the signature on the ballot $i$ using $S_i$ as the verification public key (against generator $F$), and aborts if this fails.
        \item If $S_i$ appears with any other valid ballot, discard all but the most recent such ballot, according to bulletin board ordering.
    \end{enumerate}
    \item Assembles the set of $N_{\text{valid}}$ remaining valid ballots, now indexed by $i$.
    \item For each choice index $l \in [0,k)$:
    \begin{enumerate}
        \item Computes the ballot sums for choice $l$ by setting \[ \overline{D}_l = \sum_{i=0}^{N_{\text{valid}}-1} D_{i,l} \] and \[ \overline{E}_l = \sum_{i=0}^{N_{\text{valid}}-1} E_{i,l}. \]
        \item Finalizes the decryption \[ \func{VerifyDecrypt}(\overline{D}_l, \overline{E}_l, \{j, Y_j, R_{l,j}^{\text{dec}}, \Pi_{l,j}^{\text{dec}}\}_{j=1}^t) \mapsto t_l \] to obtain the total votes $t_l$ for choice $l$, and aborts if this fails.
    \end{enumerate}
\end{enumerate}


\section{Remarks}

While we do not provide a formal security analysis here, we note that Aura meets our informal requirements.
All cryptographic constructions use only public parameters, and completeness properties map to overall protocol correctness.
We obtain universal verifiability since any observer can run $\func{VerifyBallot}$ on all ballots appearing on the bulletin board, and run $\func{VerifyTally}$ to check that these ballots all appear in the correct tally.
Vote privacy follows from the properties of the cryptographic primitives used in $\func{Vote}$ and $\func{Tally}$ under the assumption of no malicious threshold cohort of talliers.
Voter anonymity is asserted unconditionally by the use of a commitment set proof, and coercion resistance follows from the use of encrypted ballot serial numbers that are unique and fixed for each voter identity.


\subsection{Efficiency}

We discuss overall scaling of Aura algorithms here, and provide a more concrete comparison to ElectionGuard in Appendix \ref{app:efficiency}.

The most computationally-expensive construction in an Aura election is the commitment set membership proof associated to each ballot.
The size of this proof scales as $O(\log(N_{\text{voters}}))$ using the instantiation referenced.
While verification at first appears to scale as $O(N_{\text{voters}})$, the use of efficient multiscalar multiplication algorithms \cite{pippenger} can reduce this complexity to $O(N_{\text{voters}}/\log(N_{\text{voters}}))$ with good constants.

Further, the instantiation supports efficient batch verification.
When verifying proofs from multiple ballots, verifier weighting of common group elements in the required multiscalar multiplication evaluation makes the marginal verification complexity constant, amortizing the overall cost across the batch.
Interestingly, this use of batch verification can make overall Aura ballot verification several times more efficient than existing efficient mixnet constructions \cite{efficient_shuffle,groth_shuffle}.

Other verification steps in $\func{VerifySetup}$, $\func{VerifyBallot}$, and $\func{VerifyTally}$ imply lower complexity, or may be similarly batched for improved efficiency.

These observations make Aura a competitive candidate for suitable applications.


\bibliographystyle{splncs04}
\bibliography{main}


\appendix


\section{Bit commitment proving system}
\label{app:bit}

We now show an efficient instantiation of a bit commitment proving system.
This is a generalization of the construction used in \cite{bootle}, which in our notation supports only $w = 1$.
The proof of security follows similarly with only minor straightforward modifications, so we omit it here.

While in Aura we describe a non-interactive construction, we show here the corresponding interactive protocol, and note that the strong Fiat-Shamir technique easily applies.

\begin{enumerate}
    \item The prover selects $r_A, r_C, r_D, \{a_i\}_{i=1}^{k-1} \in \F$ uniformly at random, and sets \[ a_0 = - \sum_{i=1}^{k-1} a_i. \]
    \item The prover computes the Pedersen vector commitments
    \begin{alignat*}{1}
        A &= r_A H + \sum_{i=0}^{k-1} a_i G_i \\
        C &= r_C H + \sum_{i=0}^{k-1} a_i(1 - 2b_i) G_i \\
        D &= r_D H + \sum_{i=0}^{k-1} a_i^2 G_i
    \end{alignat*}
    and sends $A, C, D$ to the verifier.
    \item The verifier selects a challenge $x \in \F \setminus \{0\}$ uniformly at random, and sends $x$ to the prover.
    \item For each $i \in [1,k)$, the prover sets $f_i = b_i x + a_i$.
    The prover also sets $z_a = rx + r_A$ and $z_C = r_C x + r_D$, and sends $\{f_i\}_{i=0}^{k-1}, z_A, z_C$ to the verifier.
    \item The verifier sets \[ f_0 = wx - \sum_{i=1}^{k-1} f_i \] and accepts the proof if and only if the following hold:
    \begin{alignat*}{1}
        A + xB &= z_A H + \sum_{i=0}^{k-1} f_i G_i \\
        xC + D &= z_C H + \sum_{i=0}^{k-1} f_i(x - f_i) G_i
    \end{alignat*}
\end{enumerate}


\section{Comparison to ElectionGuard}
\label{app:efficiency}

We compare the efficiency of some Aura components to those of ElectionGuard \cite{electionguard}, since its design is well specified.
However, we note that ElectionGuard offloads voter anonymity to election organizers, which Aura specifically avoids through the use of voter-produced anonymity proofs; as a result, we cannot provide a direct comparison.

However, it is instructive to examine how Aura's other ballot-related proofs compare.
In particular, we note that ballot proofs in both protocols have two overall goals: they must show that each option encrypts to 0 (the voter does not choose it) or 1 (the voter chooses it) against the correct ElGamal key, and must further show that only a specified number of options are chosen.
In Aura, this is done using $\func{EncValVerify}$ on each encrypted option from a ballot, and $\func{BitVerify}$ on the sum of all encrypted options.
Specifically, the former asserts that each option is a valid encryption of \textit{some} message (secretly known by the prover) against the proper key, and the latter asserts both that each such message is in $\{0,1\}$ and that the selection limit is satisfied.
This approach succeeds in part because each choice is assocated to a unique group generator as part of the ElGamal encryptions.

In ElectionGuard, a related but different approach is taken.
One proving system uses a disjunction approach to show that each encrypted option corresponds to a message in $\{0,1\}$ against the proper key.
Another proving system shows that the sum of all such encryptions corresponds to the selection limit.
This approach succeeds in part because each choice uses the same group generators for ElGamal encryption.

As before, suppose an election has $k$ options, and that a voter must select between $k_{\text{min}}$ and $k_{\text{max}}$ of them.
Let $k' = k + k_{\text{max}} - k_{\text{min}}$ for convenience.
We show the total size of the ballot-related proofs in Aura and ElectionGuard in Table \ref{table:size}, assuming for Aura a standard Schnorr-type instantiation of the required encryption validity proving system.
In both cases, we generalize and assume all proof elements can be represented using a fixed and common size.\footnote{While Aura is presented for general groups, \cite{electionguard} specifies particular group parameters.}
Further, we account for batch verification, where it is possible to present Schnorr-type proving systems for both protocols either in a manner that supports efficient verification of multiple proofs at the same time (by including the initial prover messages in the proof), or in a manner not supporting this (by including the claimed Fiat-Shamir challenge in the proof, and requiring the verifier to reconstruct the prover messages); this affects the size of each proof and the overall computational complexity.

\begin{table}
    \centering
    \caption{Total size of each ballot's validity proofs for Aura and ElectionGuard, given in proof elements, both supporting batch verification and not}
    \label{table:size}
    \begin{tabular}{l|c|c|}
        Protocol & Size (batching) & Size (no batching) \\
        \hline
        Aura & $5k' + 4$ & $4k' + 4$ \\
        ElectionGuard & $8k' + 3$ & $4k' + 2$ \\
    \end{tabular}
\end{table}

We note an interesting tradeoff, in that the total size of a ballot varies between the two protocols depending on the need for batch verification support.
In the case where batch verification is desired or required, Aura ballots are significantly smaller than those of ElectionGuard; if batch verification is not used, Aura ballots are slightly larger.
Since Aura is particularly intended for efficient use in decentralized settings where both the size of the public bulletin board and overall verification complexity are considered limited resources, this provides a distinct and notable advantage.


\end{document}

